\documentclass[journal]{vgtc}                % final (journal style)
%\documentclass[review,journal]{vgtc}         % review (journal style)
%\documentclass[widereview]{vgtc}             % wide-spaced review
%\documentclass[preprint,journal]{vgtc}       % preprint (journal style)

%% Uncomment one of the lines above depending on where your paper is
%% in the conference process. ``review'' and ``widereview'' are for review
%% submission, ``preprint'' is for pre-publication, and the final version
%% doesn't use a specific qualifier.

%% Please use one of the ``review'' options in combination with the
%% assigned online id (see below) ONLY if your paper uses a double blind
%% review process. Some conferences, like IEEE Vis and InfoVis, have NOT
%% in the past.

%% Please note that the use of figures other than the optional teaser is not permitted on the first page
%% of the journal version.  Figures should begin on the second page and be
%% in CMYK or Grey scale format, otherwise, colour shifting may occur
%% during the printing process.  Papers submitted with figures other than the optional teaser on the
%% first page will be refused. Also, the teaser figure should only have the
%% width of the abstract as the template enforces it.

%% These few lines make a distinction between latex and pdflatex calls and they
%% bring in essential packages for graphics and font handling.
%% Note that due to the \DeclareGraphicsExtensions{} call it is no longer necessary
%% to provide the the path and extension of a graphics file:
%% \includegraphics{diamondrule} is completely sufficient.
%%
\ifpdf%                                % if we use pdflatex
  \pdfoutput=1\relax                   % create PDFs from pdfLaTeX
  \pdfcompresslevel=9                  % PDF Compression
  \pdfoptionpdfminorversion=7          % create PDF 1.7
  \ExecuteOptions{pdftex}
  \usepackage{graphicx}                % allow us to embed graphics files
  \DeclareGraphicsExtensions{.pdf,.png,.jpg,.jpeg} % for pdflatex we expect .pdf, .png, or .jpg files
\else%                                 % else we use pure latex
  \ExecuteOptions{dvips}
  \usepackage{graphicx}                % allow us to embed graphics files
  \DeclareGraphicsExtensions{.eps}     % for pure latex we expect eps files
\fi%

%% it is recomended to use ``\autoref{sec:bla}'' instead of ``Fig.~\ref{sec:bla}''
\graphicspath{{gfx/}{./}} % where to search for the images

\usepackage{microtype}                 % use micro-typography (slightly more compact, better to read)
\PassOptionsToPackage{warn}{textcomp}  % to address font issues with \textrightarrow
\usepackage{textcomp}                  % use better special symbols
\usepackage{mathptmx}                  % use matching math font
\usepackage{times}                     % we use Times as the main font
\renewcommand*\ttdefault{txtt}         % a nicer typewriter font
\usepackage{cite}                      % needed to automatically sort the references
\usepackage{tabu}                      % only used for the table example
\usepackage{booktabs}                  % only used for the table example
\usepackage{makecell}
\usepackage{multirow}

%% We encourage the use of mathptmx for consistent usage of times font
%% throughout the proceedings. However, if you encounter conflicts
%% with other math-related packages, you may want to disable it.

%% In preprint mode you may define your own headline.
%\preprinttext{To appear in IEEE Transactions on Visualization and Computer Graphics.}

%% If you are submitting a paper to a conference for review with a double
%% blind reviewing process, please replace the value ``0'' below with your
%% OnlineID. Otherwise, you may safely leave it at ``0''.
\onlineid{0}

%% declare the category of your paper, only shown in review mode
\vgtccategory{Research}
%% please declare the paper type of your paper to help reviewers, only shown in review mode
%% choices:
%% * algorithm/technique
%% * application/design study
%% * evaluation
%% * system
%% * theory/model
\vgtcpapertype{theory/model}

%% Paper title.
% James' title suggestions
% - Reproducing 'Graphical Perception' with CNNs
% - 
\title{Evaluating `Graphical Perception' with CNNs}

%% This is how authors are specified in the journal style

%% indicate IEEE Member or Student Member in form indicated below
%\author{Daniel Haehn, \textit{Member, IEEE}, James Tompkin, and Hanspeter Pfister}
\author{Daniel Haehn, James Tompkin, and Hanspeter Pfister}
\authorfooter{
%% insert punctuation at end of each item
\item Daniel Haehn, and Hanspeter Pfister are with the Paulson School of Engineering and Applied Sciences at Harvard University. \\
E-mail: \{haehn,pfister\}@seas.harvard.edu.
%
\item James Tompkin is with the Thomas J. Watson Sr. Center for Information Technology at Brown University. \\E-mail: james\_tompkin@brown.edu.
}

%other entries to be set up for journal
\shortauthortitle{Haehn \MakeLowercase{\textit{et al.}}: Evaluating `Graphical Perception' with CNNs}
%\shortauthortitle{Firstauthor \MakeLowercase{\textit{et al.}}: Paper Title}

%% Abstract section.
\abstract{%Convolutional neural networks are being successfully used for image understanding and object recognition tasks while regularly outperforming humans. Despite their tremendous success, little is known perceptional capabilities. In there Graphical Perception paper from 1984, Cleveland and McGill define elementary perceptual tasks that let people extract quantitative information from visualizations and measure human perception of different visual encodings with in depth user studies. 
%We replicate the experimental setup of Cleveland and McGill and evaluate the perceptional capabilities of four modern classifiers based on neural networks. We systematically test how the classifers perform on a) elementary perceptual tasks with increasing parametric complexity, b) the position-angle experiment which compares pie and bar charts, c) the position-length experiment which compares grouped and divided bar charts, and d) the bars and framed rectangles experiment where visual cues help to perceive information. We present the results of these experiments to foster the understanding of how such classifiers can be successfully applied to data visualizations. On this journey, we also study how the feed-forward neural networks obey Weber's law which defines the proportional relation between perceivable information and distribution size. We further introduce a ranking of elementary visual encodings targeted towards modern deep neural networks and derive practical evidence for properties of convolutional neural networks such as the translation invariance.  
Convolutional neural networks can successfully perform many computer vision tasks on images, and their learned representations are often said to mimic the early layers of the visual cortex. But can CNNs understand graphical perception for visualization? We investigate this question by reproducing Cleveland and McGill's seminal 1984 experiments, which measured human perception efficiency of different visual encodings and defined elementary perceptual tasks for visualization. We measure the graphical perceptual capabilities of four classifiers on a) elementary perceptual tasks with increasing parametric complexity, b) the position-angle experiment that compares pie charts to bar charts, c) the position-length experiment that compares grouped and divided bar charts, and d) the bars and framed rectangles experiment where visual cues aid perception. We also study how feed-forward neural networks obey Weber's law, which defines the proportional relation between perceivable information and distribution density. We present the results of these experiments to foster the understanding of how CNN classifiers succeed and fail when applied to data visualizations.
} % end of abstract

%% Keywords that describe your work. Will show as 'Index Terms' in journal
%% please capitalize first letter and insert punctuation after last keyword
\keywords{Machine Perception, Deep Learning}

%% ACM Computing Classification System (CCS). 
%% See <http://www.acm.org/class/1998/> for details.
%% The ``\CCScat'' command takes four arguments.

%\CCScatlist{ % not used in journal version
% \CCScat{K.6.1}{Management of Computing and Information Systems}%
%{Project and People Management}{Life Cycle};
% \CCScat{K.7.m}{The Computing Profession}{Miscellaneous}{Ethics}
%}

%% Uncomment below to include a teaser figure.
\teaser{
  \centering
  \includegraphics[width=\linewidth]{teaser.pdf}
  \caption{\textbf{Computing Cleveland and McGill's Position-Angle Experiment using Convolutional Neural Networks.} We replicate the original experiment by asking visual cortex inspired machine learning classifiers to assess the relationship between values encoded in pie charts and bar charts. Similar to the findings of Clevenland and McGill~\cite{cleveland_mcgill}, our experiments show that CNNs read quantities more accurately from bar charts (mean squared error, MSE in green).}
	\label{fig:teaser}
}

%% Uncomment below to disable the manuscript note
%\renewcommand{\manuscriptnotetxt}{}

%% Copyright space is enabled by default as required by guidelines.
%% It is disabled by the 'review' option or via the following command:
% \nocopyrightspace

\vgtcinsertpkg

%%%%%%%%%%%%%%%%%%%%%%%%%%%%%%%%%%%%%%%%%%%%%%%%%%%%%%%%%%%%%%%%
%%%%%%%%%%%%%%%%%%%%%% START OF THE PAPER %%%%%%%%%%%%%%%%%%%%%%
%%%%%%%%%%%%%%%%%%%%%%%%%%%%%%%%%%%%%%%%%%%%%%%%%%%%%%%%%%%%%%%%%

\begin{document}

%% The ``\maketitle'' command must be the first command after the
%% ``\begin{document}'' command. It prepares and prints the title block.

%% the only exception to this rule is the \firstsection command

\firstsection{Introduction}

\maketitle

Convolutional neural networks (CNNs) have been successfully applied to a wide range of visual tasks, most famously to natural image object recognition~\cite{krizhevsky_imagenet2012, simonyan_very_deep2014, szegedy2015}, for which some claim equivalent or better than human performance. This performance comparison is often motivated by the idea that CNNs model or reproduce the early layers of the visual cortex, even though they do not incorporate many details of biological neural networks or model higher-level abstract or symbolic reasoning~\cite{yamins2016using, hassabis2017neuroscience, human_vs_machine_vision}. While CNN techniques were originally inspired by neuroscientific discoveries, recent advances in processing larger datasets with deeper networks have been the direct results of engineering efforts. Throughout this significant advancement, researchers have aimed to understand why and how CNNs produce such high performance~\cite{goodfellow_book, deeplearning_blackbox2017}, with recent works targetting the systematic evaluation of the visual perception limits of CNNs~\cite{clevr, not_so_clevr}.

One fundamental application of human vision is in understanding data visualizations. This is a task unlike processing natural images, where real-world objects and their effects are abstracted into data and represented with visual marks. As a field, visualization catalogues and evaluates human perception of these figures, such as in the seminal \emph{graphical perception} experiments of Cleveland and McGill~\cite{cleveland_mcgill}. This work describes nine elementary perceptual reasoning tasks, such as position relative to a scale, length, angle, area, and shading density, plus orders their reasoning difficulty. But, with more research interested in the machine analysis of graphs, charts, and visual encodings, it seems pertinent to question whether CNNs are able to process these basic graphical elements, and derive useful measurements from the building blocks of information visualization.

As such, we explore whether these experiments which were performed with humans can be reproduced with CNNs. Cleveland and McGill's work in the 1980s has led to many insights for modern information visualization research such as the identification of elementary perceptional tasks or that bar charts are easier to understand than pie charts. 

In order to perform this evaluation, we parametrize different visual encodings such as the elementary perceptual tasks suggested by Cleveland and McGill~\cite{cleveland_mcgill}. We then replicate the original experimental design by defining linear regression tasks for continuous variables. However, we select three modern feature generators based on convolutional neural networks and combine them with a multilayer perceptron (MLP) to include non-linearities. We include the LeNet-5 network~\cite{lenet}, the VGG19 network~\cite{simonyan_very_deep2014}, and the Xception classifiers~\cite{xception}. While we train LeNet-5 from scratch, for VGG19 and Xception we use pre-trained imagenet~\cite{imagenet} weights to further mimic the human visual system. By also using the MLP directly without convolutional feature detection as baseline, we test four different classifiers as part of the following scenarios: a) elementary perceptual tasks with increasing parameteric complexity, b) position-angle experiment comparing pie charts to bar charts, c) position-length experiment that compares grouped and divided bar charts, and d) the bars and framed rectangles experiments where visual cues aid perception. We also investigate other properties of CNNs such as whether they obey Weber's law which defines a proportional dependency between an initial distribution and perceivable change. Our experimental setup includes repetitions, randomizations, and regularizations to prevent any bias. 

Our motivation for replicating Cleveland and McGill's experiments stems from the thought that computational perception seems to be closely related to biological vision. If human perception yields certain results, maybe we can replicate these results with machine vision. As our first contribution, we study the elementary perceptual tasks by Cleveland and McGill and then systematically parametrize and evaluate them for computational perception with our four classifiers. This setup includes also cross-network evaluations which give insight into the generalizability of the classifiers. It also yields our second contribution: a ranking of Cleveland and McGill's elementary perceptual tasks for our tested CNN architectures. Further, we replicate the \emph{position-angle} and the \emph{position-length} experiments which contributes to the general knowledge of superior perceivability of bar charts to pie charts in certain conditions. We then reproduce the \emph{bars-and-framed-rectangles} experiment with our classifiers. Here, we also include an additional experiment to test the Weber-Fechner's law for CNNs. Our experiments yield a TODO. Finally, we discuss our findings and derive recommendations for allowing CNNs to perceive visualizations.

We accompany this paper and detailed supplemental material with open source code~\footnote{Code, data, results and more are available at: \url{http://rhoana.org/perception}}, data, and results to enable a framework for the development and evaluation of new network architectures for graphical perception.


%\emph{Can we leverage decades of visualization research to understand the way convolutional neural networks process data?}
%
%Information visualization has been an established research field for decades which has resulted in numerous insightful findings in regards to how human beings can best process information visually.
%Unpublished preliminary experiments have shown that data representations customized for the human eye also can improve the performance of an automatic classifier. 
%While the reasons for this are still unknown, specified research can most certainly advance the understanding of deep neural networks.
%
%\subsection{From Biological Vision to Machine Learning}
%
%%\begin{figure}[t]
%%	  \includegraphics[width=\linewidth]{biology_vs_cnn.png}
%%  \caption{The Biological Vision (schematic)}
%%	\label{fig:vision}
%%\end{figure}
%
%Biological vision is an extremely powerful system which allows humans the ability, and seemingly without effort, to recognize an enormous amount of distinct objects in the world. 
%Object detection is extremely difficult and therefore is especially impressive as light intensities can change by levels of magnitude and contrast between foreground and background is so often low. 
%In addition, the visual scene changes every time the human body or human eyes move. 
%This visual system exhibits a very noisy structure but because it is organized by layers it has inspired the mathematical theory of multilayer neural networks. 
%What is remarkable is that even though current machine learning models do not resemble the complexity of its biological pendant, they inherently generalize extremely well. 
%Neural networks trained on one specific task can be used to perform detection or segmentation of, seemingly, unrelated objects with relatively minor retraining. 
%The reported classification performance is superior to that of humans and the question in regards to their functionality opens an interesting research topic.
%
%In 1962 Hubel and Wiesel were the first to begin studying the visual cortex from the standpoint of a neuroscientist. Their experimental findings on cats and macaque monkeys suggested a hierarchy of cells with increasing complexity which was then later transferred to the hierarchical model of different layers. Twenty years later, this insight was translated to the Neocognitron quantitative model, by Fukushima and Miyake, which ultimately led to the important work of Hinton, Bengio, and LeCun in the 1980s. Their work on stochastic gradient descent approximation, and the availability of faster computer hardware then led to today’s breakthrough of deep learning networks. In the last decade, this field has exhibited rapid growth, constant evolution, and new applications in various domains.
%
%We think that the biological inspiration of modern convolutional neural networks yields the evaluation of principles of human perception with computers.
%



\section{Related Work}
\cite{linegraph_vs_scatterplot} \cite{cleveland_mcgill}

Pineo et al.~\cite{Pineo2012} present a method to automatically evaluating and optimizing visualizations using a computational model of human vision, based on a neural network simulation of the early perceptual processing in the retina and primary visual cortex. [JT] Copied from their abstract.



\section{Experimental Setup}

\change{
We compare CNNs to human baselines across five experiments:
\begin{enumerate}[label=E\arabic*.,itemsep=0.5pt, topsep=1pt, parsep=0.5pt]
    \item We use Cleveland and McGill's elementary perceptual tasks to directly estimate quantities for visual marks (position, length, direction, angle, area, volume, curvature, and shading) (Section~\ref{sec:elementary}).
    \item We reproduce Cleveland and McGill's position-angle experiment that compares pie charts to bar charts (Section~\ref{sec:positionangle}).
    \item We reproduce Cleveland and McGill's position-length experiment that compares grouped and divided bar charts (Section~\ref{sec:positionlength}).
    \item We assess the bars and framed rectangles setting from Cleveland and McGill, where visual cues aid perception (Section~\ref{sec:barsframedrectangles}). 
    \item We conduct a Weber's law point cloud experiment (Section~\ref{sec:weberslaw}).
\end{enumerate} 
}

First, we describe the commonalities across all our experiments. In each, we measure whether different CNNs can predict values from low-level visual marks. \change{We formulate these measurement tasks as logistic regression rather than classification problems, so that we can estimate continuous variables such as directions and angles.} Given a stimuli image, the networks must estimate the single quantity present or the ratio between multiple quantities present.

For each experiment, we use a single factor between-subject design, with the factor being the network used. This lets us evaluate whether different network designs are competitive against human perception results. We train each network in a supervised fashion with a mean squared error (MSE) loss between the ground-truth labels and the network's estimate of the measurement from observing the generated stimuli images. Then, we test each network's ability to generalize to new examples with separate test data, created using the same stimuli generator function but with unseen ground-truth measurement labels. % (Section~\ref{sec:data}).

%This means that not only the stimuli are distinct for train and test sets but also the associated labels (Section~\ref{sec:data}).

% We define a series of hypotheses prior to each experiment.

\subsection{Networks}
\label{sec:networks}

\noindent{\textbf{Multilayer Perceptron.}} As a baseline, we use a multilayer perceptron (MLP). \change{The MLP does not have the convolutional layers which help CNNs solve visual tasks, and so we include it to tests whether a CNN is really needed to solve our simple graphical perception tasks} (Fig.~\ref{fig:classifiers}). Our MLP contains a layer of $256$ perceptrons that are activated as Rectified Linear Units (ReLU). We train this layer with dropout (probability $= 0.5$) to prevent overfitting, and then combine these ReLU units to regress our output measurement.
\\~\\
\noindent{\textbf{Convolutional Neural Networks.}} \change{We test three CNN architectures of increasing sophistication. Each has more layers than the last, which is an indicator for the network's capacity to hierarchically represent information. Each network also has more trainable parameters than the last, which is an indicator for how much information the network is able to learn overall. While larger networks need more data to train, we expect them to perform better than simpler networks.}

\change{
Our smallest CNN is the traditional LeNet-5 with 2 layers, which was designed to recognize hand-written digits~\cite{lenet}. Next, we use the VGG19 network with 16 layers, which achieved 90\% top 5 performance in the 1000-class ImageNet object recognition challenge in 2014~\cite{simonyan_very_deep2014}. Finally, we use the Xception network with 36 layers~\cite{xception}, which achieved 95\% top 5 performance on ImageNet in 2017 and was also designed to solve the 17,000-class JFT object recognition challenge~\cite{Hinton2015}. Xception includes state-of-the-art architecture elements: residual blocks to allow it to be deeper, and depth-wise separable convolutions (or Inception blocks) to separate spatial from cross-channel correlations for more efficient parameter use. All three networks have as their last layers an MLP architecture equivalent to our baseline, and so they act as earlier image and feature processors for this final regressor. Table~\ref{tab:parameters} lists the number of trainable parameters per network.}

For \emph{VGG19} and \emph{Xception}, we train all network parameters on elementary perceptual tasks (\emph{from scratch}); and we use network parameters previously trained on the ImageNet object recognition challenge but retrain the parameters in the final MLP layers (\emph{fine tuning}). \change{We know that humans are able to perform graphical perception tasks, and so maybe these pre-trained parameters that mimic early-layer human vision features are useful for the task. However, parameters trained from scratch are unlikely to mimic human features, as the network has only seen visualization tasks and not natural images (i.e., growing up not in Flatland \cite{abbott1885flatland}, but in Visland).}
\\~\\
\noindent{\textbf{Optimization.}} All hyperparameters, optimization methods, and stopping conditions are fixed across networks (Table \ref{tab:parameters}). We train for $1000$ epochs using stochastic gradient descent with Nesterov momentum but stop early if the validation loss does not decrease for ten epochs. \change{Each epoch trains the network upon every stimuli, with model updates after every mini-batch of 32 stimuli.}
\\~\\
\noindent{\textbf{Environment.}} We run all experiments on NVIDIA Titan X and Tesla V100 GPUs. We use Python scikit-image to generate the stimuli and use Keras with TensorFlow to train the networks. % and scikit-learn libraries - JT: What do we use this for?

\begin{table}[t]
\centering
\caption{\textbf{Network Training.} We use different CNN feature generators as input to a multilayer perceptron, which results in different sets of trainable parameters. As a baseline, we also train the MLP directly. Optimization conditions are fixed across networks and experiments.}
\resizebox{\linewidth}{!}{
\begin{tabular}{lr}
%	\toprule
%	\makecell{Classifier} & \makecell{Convolutional\\Layers} & \makecell{Trainable\\Parameters} \\
%	\midrule
%	MLP & $0$ & $2,560,513$ \\
%	\emph{LeNet} + MLP & $2$ & $8,026,083$ \\
%	\emph{VGG19} + MLP & $16$ & $21,204,545$ \\
%	\emph{Xception} + MLP & $36$ & $25,580,585$ \\
%	\bottomrule
	\toprule
	Network & \makecell{Trainable\\Parameters} \\
	\midrule
	MLP & $2,560,513$ \\
	\emph{LeNet} + MLP & $8,026,083$ \\
	\emph{VGG19} + MLP & $21,204,545$ \\
	\emph{Xception} + MLP & $25,580,585$ \\
	\bottomrule
\end{tabular}
\begin{tabular}{lr}
\toprule
\multicolumn{2}{l}{Optimization (SGD)} \\
\midrule
Learning rate & $0.0001$ \\
Momentum & Nesterov \\
\hspace{0.2cm} Value & $0.9$ \\
Batch size & 32 \\
Epochs & $1000$ (Early stop) \\
\bottomrule
\end{tabular}
}
\label{tab:parameters}
\vspace{-4mm}
\end{table}
%\begin{figure}[t]
%	\centering
%	  \includegraphics[width=\linewidth]{classifiers.pdf}
%  \caption{The multilayer perceptron (MLP) in our experiments has 256 neurons which are activated as rectified linear units (ReLU). We use Dropout regularization to prevent overfitting. We learn categorical and unordered dependent variables using the softmax function and perform linear regression for continuous variables. The MLP can learn the visualizations directly but we also learn features generated by LeNet (2 conv. layers, filter size $5\time5$), VGG19 trained on ImageNet (16 conv. layers, filter size $3\times3$), or Xception trained on ImageNet (36 conv. layers, filter size $3\times3$) to increase the number of trainable parameters.}
%	\label{fig:classifiers}
%\end{figure}
\begin{figure}[t]
	\centering
	
    \subfloat[Feature Generation]{
		\includegraphics[width=4.8cm,valign=c]{classifier_left.pdf}
		\vphantom{\includegraphics[width=3.5cm,valign=c]{classifier_right.pdf}}
	}
	\hfill
    \subfloat[Multilayer Perceptron]{
		\includegraphics[width=3.5cm,valign=c]{classifier_right.pdf}
	}

  \caption{\textbf{Network Architecture.} We use a multilayer perceptron (MLP) to perform linear regression for continuous variable output. We also learn convolutional features through LeNet (2 layers, filter size $5\times5$), VGG19 (16 layers, filter size $3\times3$), or Xception (36 layers, filter size $3\times3$) to test different model complexities.}
  \label{fig:classifiers}
  \vspace{-0.5cm}
\end{figure}

\subsection{Data}
\label{sec:data}

\noindent\textbf{Image Stimuli and Labels.} 
We create our stimuli as 100$\times$100 binary images, rasterized without interpolation. We develop a parameterized stimuli generator for each elementary task, with the number of possible parameter values differing per experiment (we summarize these in Table 1 of the supplemental material). Before use, we scale the generated images to the range of $-0.5$ to $0.5$ for value balance. Then, we add 5\% random noise (uniformly distributed between $-0.025$--$0.025$) to each pixel to introduce variation that prevents the networks from simply `remembering' each individual image. \change{In supplemental material Section 2, we visually compare how our stimuli vary from Cleveland and McGill's original stimuli for E1--5, and justify any differences.}

Each stimuli image also has an associated ground truth label representing the parameter set that generated the image.
%, e.g., the length in pixels of a bar. As before, 
We scale these labels to the range of $0.0$ to $1.0$ and normalize to the maximum and minimum value range for each parameter.
%, which represent the maximum and minimum values that this parameter can take.
\\~\\
\noindent\textbf{Training/Validation/Test Splits.} For each task, we use 60,000 training images, 20,000 validation images, and 20,000 test images. To create these datasets, we generate stimuli from random parameters and add them to the sets until the target number is reached, while maintaining distinct (random) parameter spaces for each set to ensure that there is no leakage between training and validation/testing.

\subsection{Measures and Analysis}
\label{sec:measuresandanalysis}

\noindent\textbf{Cross Validation.} For experiment reproducibility, we perform repeated random sub-sampling validation, also known as Monte Carlo cross-validation~\cite{xu2001monte}. We run every experiment separately twelve times (four times for the `from scratch' networks due to significantly-longer training times), and randomly select (without replacement) the $60\%$ of our data as training data, $20\%$ as validation, and $20\%$ as test. 
%Our large data sample size of 100,000 guarantees that every single observation of our parameterizations will be selected at least once (excluding noise patterns). -> JT: No it doesn't.
% Finally, we average the results over the runs. -> JT: Fine, but we'll do distributions anyway, so not worth saying.
\\~\\
\noindent{\textbf{Task Accuracy.}} In their 1984 paper, Cleveland and McGill use the midmean logistic absolute error metric (\emph{MLAE}) to measure perception accuracy. To allow comparison between their human results and our machine results, we also use MLAE for presentation:
\begin{equation}
	\textnormal{MLAE} = log_2( | \textnormal{predicted percent} - \textnormal{true percent} | + .125)
\end{equation}
In addition to this metric, we also calculate standard error metrics such as the mean squared error (\emph{MSE}) and the mean absolute error (\emph{MAE}). This allows a more direct comparison of percent errors. Please note that our networks were trained using MSE loss and not directly with MLAE.
\\~\\
\noindent{\textbf{Error Distributions/Confidence Intervals.}} \change{We check for normality in our error distributions using the D'Agostino-Pearson test: 1.14\% of our networks did not pass. These were typically from the smaller MLP or LeNet networks (see supplemental material for example error distributions). As such, we broadly assume normality of errors and follow Cleveland and McGill in presenting $95\%$ confidence intervals, computed via bootstrapping (with 10,000 rather than 1,000 samples for a more accurate estimate)}. %We approximate the value of the $97.5$ percentile point of the normal distribution for simplicity with $1.96$ as suggested by the central limit theorem~\cite{central_limit}.
\\~\\
\noindent{\textbf{Confirmatory Data Analysis.}} To accept or reject our hypotheses under this normality, we analyze dependent variables using analysis of variance (ANOVA) followed by parametric tests. % this is clear from the notation t_test... no need to mention it otherwise
\\~\\
\noindent{\textbf{Training Efficiency.}} We use the training convergence rate as a measure of how easy or hard a particular task is for the network to solve. This is defined as the MSE loss decrease per training epoch, which is an indicator of the training efficiency of the network with respect to the visual encoding. Lower MSE values are better. % and show that the network learned the task.
\\~\\
\noindent\textbf{Network Generalizability.} With sufficient capacity of trainable parameters, it is often said that a network can `memorize' the images if the data set has a low variability. Therefore it is important to consider this variability when evaluating different networks with fixed numbers of trainable parameters (Table~\ref{tab:parameters}). As discussed, we add noise to each stimulus image to increase variability. We also evaluate generalizability by asking a network previously trained for one task parameterization to answer questions about the same type of task stimuli but with more variability, e.g., estimating bar length without and with changes in stroke width.

Further, some experiments compare different visual encoding types, e.g., bar plot vs.~stacked bar plot. We train and evaluate individual networks for each task, and we also train and evaluate networks with stimuli from across the encoding types. These single decision-making \change{networks} better mimic judgments that a human would be able to make. 
%This scenario affects the optimization process and result in networks with more flexible knowledge with the caveat of longer training times. -> JT: Save it for the results discussion.

\subsection{Human Baselines}
\change{
We take human baseline measurements for the position-angle (E2) and position-length (E3) experiments from Cleveland and McGill~\cite{cleveland_mcgill}, which had 51 participants. For the position-length experiment, we are also able to take human baseline measurements from Heer and Bostock's crowdsourced reproduction of Cleveland and McGill's experiments~\cite{HeerBostock2010}, which had 50 participants. Each participant in both experiments reviewed 10 stimuli for each condition.

For the three remaining experiments (E1,E4,E5), we use Amazon Mechanical Turk to crowdsource new human baselines from 25 participants. Each participant was shown 10 stimuli for each experiment condition (nine for E1, two for E4, and three for E5), with three stimuli per condition presented as practice stimuli. This totaled 182 HITs per participant, with each HIT worth \$0.06. Average HIT time was 27 seconds. 85 HITs total were rejected for out of range values. Participants were recruited from the US, with Master Worker or better qualification. As in Cleveland and McGill, participants were requested to perform ``a quick visual judgment and not try to make precise measurements, either mentally or with a physical object such as a pencil or your finger.''
}

% !TeX root = paper.tex
\section{Experiment 1: Elementary Perceptual Tasks}
\label{sec:elementary}

Cleveland and McGill describe a set of elementary graphical perceptual tasks across ten encodings, where each encodes a quantitative variable in a graphical element or visual mark~\cite{cleveland_mcgill,cleveland1985graphical}. These tasks are the low-level building blocks for information visualizations (Figure~\ref{fig:figure1_results}): %(Table~\ref{tab:encoding_parameters}):
estimating position on a common scale, position on non-aligned scales, length, direction (or slope), angle, area, volume, curvature, and shading (or ink density). We leave color saturation experiments for future work.

For these tasks, we create visualizations as 100$\times$100 raster images, and test whether each of our networks is able to regress quantities from the images. We generate multiple versions of each elementary perceptual task, which allows us to increase task complexity. For instance, for \emph{Position Common Scale}, first we only vary the $y$-position of the spot to estimate against the scale, then we include translation along the $x$-axis, and then we vary the spot size. Each variation increases the size of the space of possible images for the network to predict (Table 1; supplemental material). Since empirical evidence suggests that CNNs can interpolate between different training data points, we expect the networks to perform on variations of a similar perceptual task.


%Cleveland and McGill did not explicitly test human perception of single instances of these encodings. -> JT: We now know this isn't true.

%\begin{figure}[t]
%	  \includegraphics[width=\linewidth]{figure1_overview.pdf}
%  \caption{\textbf{Elementary Perceptual Tasks.} Rasterized visualizations of the elementary perceptual tasks as defined by Cleveland and McGill~\cite{cleveland_mcgill} (color saturation excluded). We vary the parameters of each perceptual task and then assess the interpretability of feed-forward neural networks.}
%	\label{fig:elementary_perceptual_tasks}
%\end{figure}



% \begin{table}[t]
% \centering
% \caption{\textbf{Elementary Perceptual Tasks.} Rasterized visualizations of our elementary perceptual tasks as defined by Cleveland and McGill~\cite{cleveland_mcgill} (color saturation excluded). We sequentially increase the number of parameters for every task (e.g., by adding translation). This introduces variability and creates increasingly more complex datasets.}
% \resizebox{0.9\linewidth}{!}{
% \begin{tabular}{lllr}
% 	\toprule
% 	\multicolumn{2}{l}{Elementary Perceptual Task} & ~ & Permutations\\
% 	\midrule
% 	\raisebox{-.85\height}{\includegraphics[width=.5in]{position_common_scale.pdf}} & \makecell[tl]{\emph{Position Common Scale}\\~~~Position Y\\~~~+ Position X \\~~~+ Spot Size \\} &~& \makecell[tr]{~\\ $60$ \\ $3,600$ \\ $21,600$}\\

% 	\midrule
% 	\raisebox{-.85\height}{\includegraphics[width=.5in]{position_non_aligned_scale.pdf}} & \makecell[tl]{\emph{Position Non-Aligned Scale}\\~~~Position Y\\~~~+ Position X \\~~~+ Spot Size \\} &~& \makecell[tr]{~\\ $600$ \\ $36,000$ \\ $216,000$}\\

% 	\midrule
% 	\raisebox{-.95\height}{\includegraphics[width=.5in]{length.pdf}} & \makecell[tl]{\emph{Length}\\~~~Length\\~~~+ Position Y \\~~~+ Position X \\~~~+ Width} &~& \makecell[tr]{ ~\\$60$ \\ $2,400$ \\ $144,000$\\$864,000$}\\

% 	\midrule
% 	\raisebox{-.85\height}{\includegraphics[width=.5in]{direction.pdf}} & \makecell[tl]{\emph{Direction}\\~~~Angle\\~~~+ Position Y \\~~~+ Position X} &~& \makecell[tr]{ ~\\$360$ \\ $21,600$ \\ $1,296,000$}\\

% 	\midrule
% 	\raisebox{-.85\height}{\includegraphics[width=.5in]{angle.pdf}} & \makecell[tl]{\emph{Angle}\\~~~Angle\\~~~+ Position Y \\~~~+ Position X} &~& \makecell[tr]{ ~\\$90$ \\ $5,400$ \\ $324,000$}\\

% 	\midrule
% 	\raisebox{-.85\height}{\includegraphics[width=.5in]{area.pdf}} & \makecell[tl]{\emph{Area}\\~~~Radius\\~~~+ Position Y \\~~~+ Position X} &~& \makecell[tr]{ ~\\$40$ \\ $800$ \\ $16,000$}\\

% 	\midrule
% 	\raisebox{-.85\height}{\includegraphics[width=.5in]{volume.pdf}} & \makecell[tl]{\emph{Volume}\\~~~Cube Sidelength\\~~~+ Position Y \\~~~+ Position X} &~& \makecell[tr]{ ~\\$20$ \\ $400$ \\ $8,000$}\\
	
% 	\midrule
% 	\raisebox{-.85\height}{\includegraphics[width=.5in]{curvature.pdf}} & \makecell[tl]{\emph{Curvature}\\~~~Midpoint Curvature\\~~~+ Position Y \\~~~+ Position X} &~& \makecell[tr]{ ~\\$80$ \\ $1,600$ \\ $64,000$}\\	

% 	\midrule
% 	\raisebox{-.85\height}{\includegraphics[width=.5in]{shading.pdf}} & \makecell[tl]{\emph{Shading}\\~~~Density\\~~~+ Position Y \\~~~+ Position X} &~& \makecell[tr]{ ~\\$100$ \\ $2,000$ \\ $40,000$}\\	
% %	
% 	\bottomrule
% \end{tabular}
% }
% \label{tab:encoding_parameters}
% \vspace{-0.25cm}
% \end{table}



%
\subsection{Hypotheses}

\begin{hypolist}
\item \textbf{H1.1} \textbf{The CNNs tested will be able to regress quantitative variables from graphical elements.} We generate different visual encodings %(Table~\ref{tab:encoding_parameters})
and test whether the CNNs can measure them. %, and relate the results to accuracies obtained by humans on similar tasks.

\item \textbf{H1.2} \textbf{CNN perceptual performance will depend on network architecture.} We evaluate multiple regressors with different numbers of trainable parameters. We expect a more complex network (with more trainable parameters) to perform better on elementary perceptual tasks than a network with less complexity.

\item \textbf{H1.3} \textbf{Some visual encodings will be easier to learn than others for the CNNs tested.} Cleveland and McGill order the elementary perceptual tasks by accuracy. We expect this order to be relevant for computing graphical perception.

\item \textbf{H1.4} \textbf{Networks trained on perceptual tasks will generalize to more complex variations of the same task.} Empirical evidence suggests that CNNs can generalize between different training data points. We create visual representations of the elementary perceptual tasks with different variability and expect that networks will be able to generalize to slight task variations.
\end{hypolist}

%	
%	We suspect that CNNs can 'learn` absolute quantities encoded using low-level visual 
%	
%	While much simpler models than their biological pendant, convolutional neural networks are heavily influenced by our biological knowledge of the visual system. Such classifiers therefor follow the same principles as human perception.

% \begin{figure}[tbhp]
% 	\centering
% 	\includegraphics[width=\linewidth]{figure1_slim_only_last.pdf}
% %	\includegraphics[width=0.48\linewidth]{figure1_slim_right.pdf}
% 	\caption{\textbf{Elementary perceptual tasks results for most complex task parameterization.} \emph{Left:} Example stimuli image. \emph{Right:} MLAE and 95\% confidence intervals for different networks. Lower MLAE scores are better. The * indicates ImageNet networks instead of being trained from scratch.}
% 	\label{fig:figure1_results}
% %	\vspace*{1in}
% \end{figure}
\begin{figure*}[t]
	\centering
	\includegraphics[width=.445\linewidth]{figure1_slim_only_last_NEW_LEFT.pdf}
	\includegraphics[width=.45\linewidth]{figure1_slim_only_last_NEW_RIGHT_with_multi.pdf}
%	\includegraphics[width=0.48\linewidth]{figure1_slim_right.pdf}
	\caption{\textbf{Elementary perceptual tasks results for the most complex task parameterization.} In each column: \emph{Left:} Example stimuli image. \emph{Right:} MLAE and \change{bootstrapped} 95\% confidence intervals for different networks. Lower MLAE scores are better. The * indicates fine-tuned ImageNet weights instead of weights trained from scratch. \change{Bottom right shows `multi' VGG19 and Xception networks trained on all perceptual tasks, combined with optional $9\times$ increase of training data.} }
	\label{fig:figure1_results}
%	\vspace*{1in}
\end{figure*}

\subsection{Results}

\change{Midmean random performance for all tasks is $MLAE=4.8$, save for direction ($4.6$) where the 0--360 wrap improves random performance.} % JT check this!
% These values are similar or better than the range of MLAE=$3.02$--$3.82$ ($8-14\%$ error) for a similar experiment by Cleveland and McGill testing human estimation of relations between elementary perceptual tasks~\cite{cleveland1985graphical}. 

%While the performance varies for the different encodings, we observe for all networks performance similar to the measured human performance by Cleveland and McGill~\cite{cleveland1985graphical}. These results confirm our initial hypotheses.
\noindent{\textbf{Overall Accuracy.}} 
The tested CNNs and MLP can regress the visually-encoded quantities in most cases (Fig.~\ref{fig:figure1_results}), with average error across all classifiers and tasks as \textit{MLAE}$=1.598$ ($SD=0.392$) and \textit{MAE}=$2.903$ ($SD=0.845$). 
Based on these results, we \textbf{accept H1.1}.
\\~\\
\noindent{\textbf{Comparing Networks.}} 
Across network architectures and training schemes, there is considerable difference in performance. In order of decreasing error: 
The MLP has \textit{MLAE}$=2.943$ ($SD=0.857$), 
for LeNet $2.125$ ($SD=0.38$), 
Xception trained on ImageNet $1.627$ ($SD=0.462$), 
Xception trained from scratch $1.511$ ($SD=0.485$),
VGG19 trained on ImageNet $0.979$ ($SD=0.581$), 
and VGG19 trained from scratch $0.404$ ($SD=0.407$). Overall, VGG19 performs best.

Across tasks, we compare the average regression performances for our networks and report the effect as statistically significant ($F_{5,48}=20.470,p<0.01$). Post hoc comparisons show that the differences between LeNet and the VGG19 network, independent of the used weights, are significant ($t_{16}=4.674,p<0.01$ and $t_{16}=8.746,p<0.01$). VGG19 from scratch and Xception (both versions) perform significantly differently, with Xception from scratch ($t_{16}=4.944,p<0.01$) and Xception with ImageNet weights ($t_{16}=5.621,p<0.01$). However, differences between LeNet and both Xception networks are not significant. Taken collectively, we \textbf{partially accept H1.2}, in that higher network complexity does not automatically infer greater performance.
\\~\\
\noindent{\textbf{Ranking of Visual Encodings.}} Cleveland and McGill provide an ordering of elementary visual encodings based on theoretical arguments and experimental results. We compare their ranking with rankings of our networks in Table~\ref{tab:ranking}. Overall, there is significant variability in the rankings between architectures (Fig.~\ref{fig:figure1_results}). Area estimation is an easier task for all networks, while direction and angle estimation are more difficult. It is harder to distinguish differences between position, length, curvature, and shading tasks. Further, the volume task suffers high variability in performance across cross-validation splits, which suggests that the image noise affects the outcome more than for other tasks. 

We note that the number of permutations across tasks does not strictly relate to network performance. While area in its most complex parameterization has 16,000 permutations, and so should be easier to learn, length has 864,000, yet VGG19 is able to achieve similar performance for both tasks. Likewise, direction has 4$\times$ more permutations than angle, yet the networks achieve similar performance. %\JT{This is because...}

In sum, we \textbf{partially accept H1.3}. Further, the rankings between networks using ImageNet weights are identical, suggesting that the information about elementary perceptual tasks gained from natural images is similar given a sufficiently-complex network.
\\~\\
\noindent{\textbf{Cross-network Variability and Network Generalizability.}} We measure regression performance across networks trained with different parameterizations of the elementary perceptual tasks (Fig.~\ref{fig:cross_network}). For our best performing network (VGG19 trained from scratch), we observe that accuracy decreases only slightly as the parameterization becomes more complex if training examples expressing all variability are included (diagonal entries in each matrix).  However, VGG19 is unable to generalize to added translation or stroke width variations in the encodings, leading to increases in error. As such, we \textbf{reject H1.4}. These findings suggest that slight variations in visual encodings can confuse CNNs, making it difficult to generalize the measurement of quantities to unseen examples dissimilar from the training data.

%\JT{Discuss multi result; discuss more data result; discuss human performance.}

%Observing Figure~\ref{fig:figure1_results}, -> JT: we should do this.

%Since such a ranking see,s to heavily depend on the used regressor, the weights, and the initialization, we do not think a generalized ranking can be created and \textbf{reject H1.3}.

\begin{table}[tb]
\centering
\caption{\textbf{Elementary Perceptual Task Ranking.} We report midmean logistic absolute errors (MLAE) for each network averaged across multiple runs on the most complex parametrization of each task. The lower MLAE, the better (negative values are the best). For human performance, we report the ranking of Cleveland and McGill~\cite{cleveland_mcgill}. VGG19 performs best overall, while VGG19 * and Xception * networks using ImageNet weights yield identical rankings.}
\resizebox{\linewidth}{!}{
\begin{tabular}{cllllll}
\toprule
Human (CMcG) & MLP & LeNet & VGG19 * & \textbf{VGG19} & Xception * & Xception \\
\midrule
\multicolumn{7}{l}{\emph{Position common scale}} \\
1. & 7. (3.84) & 2. (1.36) & 5. (1.02) & \textbf{3 (-0.04)} & 5. (1.65) & 2. (1.04) \\
\multicolumn{7}{l}{\emph{Position non-aligned scale}} \\
2. & 6. (3.61) & 1. (1.35) & 6. (1.09) & \textbf{5 (0.26)} & 6. (1.71) & 1. (1.02) \\
\multicolumn{7}{l}{\emph{Length}} \\
3. & 1. (1.99) & 8. (3.19) & 4. (0.87) & \textbf{2 (-0.14)} & 4. (1.59) & 3. (1.11) \\
\multicolumn{7}{l}{\emph{Direction}} \\
3. & 9. (4.65) & 7. (3.07) & 9. (2.84) & \textbf{9 (0.92)} & 9. (3.46) & 6. (1.57) \\
\multicolumn{7}{l}{\emph{Angle}} \\
3. & 8. (4.61) & 9. (3.33) & 8. (2.31) & \textbf{7 (0.66)} & 8. (2.60) & 7. (1.69) \\
\multicolumn{7}{l}{\emph{Area}} \\
4. & 2. (2.01) & 5. (2.21) & 1. (0.49) & \textbf{1 (-0.17)} & 1. (0.80) & 5. (1.38) \\
\multicolumn{7}{l}{\emph{Volume}} \\
5. & 4. (2.38) & 4. (1.91) & 7. (1.16) & \textbf{8 (0.87)} & 7. (2.03) & 9. (2.10) \\
\multicolumn{7}{l}{\emph{Curvature}} \\
5. & 3. (2.34) & 3. (1.81) & 2. (0.71) & \textbf{6 (0.28)} & 2. (1.17) & 4. (1.13) \\
\multicolumn{7}{l}{\emph{Shading}} \\
6. & 5. (3.04) & 6. (2.23) & 3. (0.73) & \textbf{4 (0.14)} & 3. (1.57) & 8. (1.82) \\

% 
%\makecell[tl]{\emph{Position}\\~~\emph{Non-aligned Scale}} & 1 & 1 & 2 & 3 & \textbf{4} & 5 & 6 \\
%\makecell[tl]{\emph{Length}} & 1 & 1 & 2 & 3 & \textbf{4} & 5 & 6 \\
%\makecell[tl]{\emph{Direction}} & 1 & 1 & 2 & 3 & \textbf{4} & 5 & 6 \\
%\makecell[tl]{\emph{Angle}} & 1 & 1 & 2 & 3 & \textbf{4} & 5 & 6 \\
%\makecell[tl]{\emph{Area}} & 1 & 1 & 2 & 3 & \textbf{4} & 5 & 6 \\
%\makecell[tl]{\emph{Volume}} & 1 & 1 & 2 & 3 & \textbf{4} & 5 & 6 \\
%\makecell[tl]{\emph{Curvature}} & 1 & 1 & 2 & 3 & \textbf{4} & 5 & 6 \\
%\makecell[tl]{\emph{Shading}} & 1 & 1 & 2 & 3 & \textbf{4} & 5 & 6 \\
%\begin{tabular}{ll}
%	\toprule
%	Task & \begin{tabular}{ccccccc}
%			Human & MLP & LeNet & VGG19 * & VGG19 & Xception * & Xception
%			\end{tabular}\\
%	\midrule
%	\raisebox{-.85\height}{\includegraphics[width=.5in]{position_common_scale.pdf}} & \makecell[tl]{\emph{Position Common Scale}\\ \begin{tabular}{ccccccc}
%
%\end{tabular}} \\
%
%	\midrule
%	\raisebox{-.85\height}{\includegraphics[width=.5in]{position_non_aligned_scale.pdf}} & \makecell[tl]{\emph{Position Non-Aligned Scale}\\~~~Position Y\\~~~+ Position X \\~~~+ Spot Size \\} &~& \makecell[tr]{~\\ $600$ \\ $36,000$ \\ $216,000$}\\
%
%	\midrule
%	\raisebox{-.95\height}{\includegraphics[width=.5in]{length.pdf}} & \makecell[tl]{\emph{Length}\\~~~Length\\~~~+ Position Y \\~~~+ Position X \\~~~+ Width} &~& \makecell[tr]{ ~\\$60$ \\ $2,400$ \\ $144,000$\\$864,000$}\\
%
%	\midrule
%	\raisebox{-.85\height}{\includegraphics[width=.5in]{direction.pdf}} & \makecell[tl]{\emph{Direction}\\~~~Angle\\~~~+ Position Y \\~~~+ Position X} &~& \makecell[tr]{ ~\\$360$ \\ $21,600$ \\ $1,296,000$}\\
%
%	\midrule
%	\raisebox{-.85\height}{\includegraphics[width=.5in]{angle.pdf}} & \makecell[tl]{\emph{Angle}\\~~~Angle\\~~~+ Position Y \\~~~+ Position X} &~& \makecell[tr]{ ~\\$90$ \\ $5,400$ \\ $324,000$}\\
%
%	\midrule
%	\raisebox{-.85\height}{\includegraphics[width=.5in]{area.pdf}} & \makecell[tl]{\emph{Area}\\~~~Radius\\~~~+ Position Y \\~~~+ Position X} &~& \makecell[tr]{ ~\\$40$ \\ $800$ \\ $16,000$}\\
%
%	\midrule
%	\raisebox{-.85\height}{\includegraphics[width=.5in]{volume.pdf}} & \makecell[tl]{\emph{Volume}\\~~~Cube Sidelength\\~~~+ Position Y \\~~~+ Position X} &~& \makecell[tr]{ ~\\$20$ \\ $400$ \\ $8,000$}\\
%	
%	\midrule
%	\raisebox{-.85\height}{\includegraphics[width=.5in]{curvature.pdf}} & \makecell[tl]{\emph{Curvature}\\~~~Midpoint Curvature\\~~~+ Position Y \\~~~+ Position X} &~& \makecell[tr]{ ~\\$80$ \\ $1,600$ \\ $64,000$}\\	
%
%	\midrule
%	\raisebox{-.85\height}{\includegraphics[width=.5in]{shading.pdf}} & \makecell[tl]{\emph{Shading}\\~~~Density\\~~~+ Position Y \\~~~+ Position X} &~& \makecell[tr]{ ~\\$100$ \\ $2,000$ \\ $40,000$}\\	

\bottomrule
\end{tabular}
}
\label{tab:ranking}
\vspace{-0.25cm}
\end{table}




\begin{figure}[tbhp]
	\centering
	  \includegraphics[width=.85\linewidth]{cross_network_small_VGG19_position_common_scale.pdf}
	  \includegraphics[width=.85\linewidth]{cross_network_small_VGG19_length.pdf}
	  \includegraphics[width=.85\linewidth]{cross_network_small_VGG19_area.pdf}
	  \includegraphics[width=.85\linewidth]{cross_network_small_VGG19_shading.pdf}
	  \vspace{.1cm}
  \caption{\textbf{Cross-network variability for perceptual tasks.} VGG19 networks trained on one set of parametrizations (X-axis) while tested across different ones (Y-axis), for the top four performing encodings. Diagonal matrix entries represent networks trained and tested on the same parameterizations. Below diagonal entries are scenarios where the test data has more parameters than the training data; above diagonal entries have fewer. We measure the mean logistic absolute error (MLAE)---the lower the score, the better. VGG19 becomes only slightly less accurate as the parameterization becomes more complex; however, it is unable to generalize to unseen element translations as error increases rapidly. Note that all networks showed similar behavior.}
	\label{fig:cross_network}
\end{figure}




\section{Position-Angle Experiment}

The position-angle experiment was originally performed by Cleveland and McGill to measure whether humans can better perceive quantities encoded as positions or as angles~\cite{cleveland_mcgill}. The actual experiment then compares pie charts versus bar charts since these map down to elementary position and angle judgement. We create rasterized images mimicking Cleveland and McGill's proposed encoding and investigate computational perception of our four classifiers.

\begin{figure}[t]
	  \includegraphics[width=\linewidth]{figure3_overview}
  \caption{\textbf{Position-Angle Experiment.} We create rasterized visualizations of pie charts and bar charts to follow Cleveland and McGill's position-angle experiment. The experimental task involves the judgement of different encoded values in comparison to the largest encoded values. The pie chart (left) and the bar chart (right) visualize the same data point. In their paper, Cleveland and McGill report less errors using bar charts.}
	\label{fig:position_angle_experiment}
\end{figure}

\subsection{Hypotheses}

We proposed four hypotheses entering the elementary perceptual task experiment:

\begin{itemize}
	\item \textbf{H2.1} \textbf{Computed perceptual performance is better using bar charts than pie charts.} Cleveland and McGill report that position judgements are almost twice as accurate as angle judgements. This renders bar charts superior to pie charts and should also be the case for convolutional neural networks.
	\item \textbf{H2.2} \textbf{Classifiers can learn position faster than angles.} We assume that understanding bar charts is easier than understanding pie charts. We suspect that our classifiers learn encodings of positions faster than of angles resulting in more efficient training and faster convergence.
\end{itemize}



\section{Experiment: Position-Length}

The position-length experiment by Cleveland and McGill involves the perception of five types of visualizations~\cite{cleveland_mcgill}. The visualizations are either grouped bar charts or divided bar charts (Fig.~\ref{fig:figure4_mlae}. Both types of graphs can show the same information but the elementary perceptual task is different. According to the theory by Cleveland and McGill, a grouped bar chart always involves the judgement of positions along a common scale. On the other hand, a divided bar chart might require length judgements in addition. Figure~\ref{fig:figure4_mlae} shows different charts of both types on the left. As identified by Cleveland and McGill, types 1, 2, and 3 involve the judgement of positions along a common scale while types 4 and 5 involve the measure of length. This means that this experiment does not just compare grouped versus divided bar charts but rather comparing position and length judgements.

For data generation we follow the same approach as in the original experiment. We generate pairs from ten values generated using Cleveland and McGill's equation

\begin{equation}
s_i = 10\times10^{(i-1)/12},\textnormal{~~~~~}i=1,...,10,
\end{equation}

with two different values for each pair. We then generate 8 other random values in the range of 10 and 93. These boundaries were chosen because we create rasterized visualizations using $100\times100$ pixel and want to stay in frame for a convolutional filter size of $5\times5$ in LeNet. We then visually encode the ten values as type 1--5. The paired quantities get marked by a single pixel. These are our quantities of interest and the task is to estimate the ratio of the smaller to the larger. We model this task as a single value regression. For type 4, we follow Cleveland and McGill's constraint that neither the top or the bottom of the marked quantities match to force length estimations rather than position.

We model this experiment as a single value regression task and ask networks to estimate the percentage of the smaller to the larger marked quantity in each visual encoding. The network first has to find the two marked quantities, then identify the smaller one, and finally estimate the ratio in comparison to the larger quantitiy. The 8 random quantities (which should be ignored by the network) push the number of possible permutations to a massive $9.20E+16$ and creates a very challenging problem.

Finally, we also train `multi' networks which include all five types resulting in an even bigger challenge.

\subsection{Hypotheses}

We proposed two hypotheses entering the elementary perceptual task experiment:

\begin{itemize}
  \item \textbf{H3.1} \textbf{Our networks can estimate all types equally well.} A grouped bar chart involves judging a position while a divided bar chart most likely (if not type 2) requires length judgements. Our rankings of elementary perceptual tasks do not yield a strong preference for either across all networks.
  \item \textbf{H3.2} \textbf{Trained `multi' networks work as well as individual trained networks.} Convolutional neural networks have massive numbers of trainable parameters. Their complexity allows them to learn different types of visual encodings in one training session. 
\end{itemize}


%\begin{figure*}[t]
%   \includegraphics[width=\linewidth]{figure4_overview.pdf}
%   
%  \caption{\textbf{Position-Length Experiment.} (Not yet) Rasterized versions of the graphs of Cleveland and McGill's position-length experiment. The perceptual task involves comparing. the two dot-marked quantities across five different visual encodings of either grouped or divided bar charts. We evaluate which type of bar chart performs better with our neural networks as a combined classification and regression problem. The first task is to select which of the marked quantities is smaller (classification) and the second task is to specify how much smaller it is (regression).}
% \label{fig:position_length_experiment}
%\end{figure*}
%\begin{table}[h]
%\centering
%\caption{\textbf{Position-Length Experiment.} Rasterized versions of the graphs of Cleveland and McGill's position-length experiment. The perceptual task involves comparing the two dot-marked quantities across five different visual encodings of either grouped or divided bar charts. We evaluate which type of bar chart performs better with our neural networks. The two marked values are chosen from a set of ten pairs which defines the dual regression task. Since the other 8 values are chosen randomly, the parameter space for images of this experiment is massive.}
%\resizebox{\linewidth}{!}{
%\begin{tabular}{lllr}
% \toprule
% \multicolumn{2}{l}{~} & ~ & Permutations\\
% \midrule
% \raisebox{-.85\height}{\includegraphics[width=.5in]{figure4_type_1.pdf}} & \makecell[tl]{Type 1: \emph{Grouped Bar Chart}\\~~~Perceptual Task: \emph{Position}\\~ \\~ \\} &~& \makecell[tr]{~\\ $9.20E+16$}\\
%
% \midrule
% \raisebox{-.85\height}{\includegraphics[width=.5in]{figure4_type_2.pdf}} & \makecell[tl]{Type 2: \emph{Divided Bar Chart}\\~~~Perceptual Task: \emph{Position}\\~ \\~ \\} &~& \makecell[tr]{~\\ $9.20E+16$}\\
% 
% \midrule
% \raisebox{-.85\height}{\includegraphics[width=.5in]{figure4_type_3.pdf}} & \makecell[tl]{Type 3: \emph{Grouped Bar Chart}\\~~~Perceptual Task: \emph{Position}\\~ \\~ \\} &~& \makecell[tr]{~\\ $9.20E+16$}\\
% 
% \midrule
% \raisebox{-.85\height}{\includegraphics[width=.5in]{figure4_type_4.pdf}} & \makecell[tl]{Type 4: \emph{Divided Bar Chart}\\~~~Perceptual Task: \emph{Length}\\~ \\~ \\} &~& \makecell[tr]{~\\ $9.20E+16$}\\
% 
% \midrule
% \raisebox{-.85\height}{\includegraphics[width=.5in]{figure4_type_5.pdf}} & \makecell[tl]{Type 5: \emph{Divided Bar Chart}\\~~~Perceptual Task: \emph{Length}\\~ \\~ \\} &~& \makecell[tr]{~\\ $9.20E+16$}\\
%
% \bottomrule
%\end{tabular}
%}
%\label{tab:pos_length_parameters}
%\end{table}
%
%\subsection{Discussion}
%
%JT: Look at the relative difficulty of the tasks. In Cleveland and McGill, types 1-5 were post-ordered by their log error such that type 1 was easiest and type 5 was hardest. Is this still the case with our CNNs?

\subsection{Results}

\begin{figure}[!t]
  \centering
    \includegraphics[width=\linewidth]{figure4_mlae_with_multi_and_humans_all.pdf}
  \caption{\textbf{Computational results of the position-length experiment.} \textit{Left:} Rasterized visualizations of type 1--5 for divided and grouped bar charts of Cleveland and McGill's position-length experiment. \textit{Right:} MLAE and $95\%$ confidence intervals for different regressors estimating the value of marked quantities in the visualizations. The VGG19 * and Xception * networks are using ImageNet weights. The top 5 rows represent networks trained on a single encoding while the last row shows `multi' networks which were trained on a random stream of types 1--5. We visually compare against human performance from the original experiment.}
  \label{fig:figure4_mlae}
\end{figure}

\noindent\textbf{Perceptual Performance.} In the original position-length experiment, Cleveland and McGill report that types 1-5 were post-ordered by their perceptual difficulty with type 1 being the easiest and type 5 the hardest. We report the average MLAE for each type across our networks (and total 56 runs per type) as follows: 
for type 1 $MLAE= 3.956 $ ($SD= 0.274 $),
for type 2 $MLAE= 3.952 $ ($SD= 0.441 $),
for type 3  $MLAE= 4.349 $ ($SD= 0.367 $),
for type 4 $MLAE= 3.668 $ ($SD= 0.256 $), and 
for type 5 $MLAE= 3.902 $ ($SD= 0.253 $). These distributions yield significance ($F_{4,25}=2.815, p<0.05$) but post hoc comparions show that really only type 3 and type 4 differ ($t_{10}=3.406, p<0.01$). This means that the networks do not prefer a certain type on average which leads us to \textbf{partially accept H2.1} and we do not replicate the same pattern as Cleveland and McGill in their human studies even though our networks' performance is clearly worse than their human baseline.
\\~\\
\noindent\textbf{`Multi' Network Performance.} In Cleveland and McGill's original position-length experiment, humans were asked to judge visualizations of types 1-5. In the bottom of Figure~\ref{fig:figure4_mlae} we average the human performances of types 1-5 to create a `multi' human. Similarly, when training our networks, we first limit each training to one stimuli. However, a `multi' network which learns types 1-5 simultaenously is closer to the human in the original user study. 




\section{Bars and Framed Rectangles Experiment}

Visual cues can help converting graphical elements back to their real world variables. Cleveland and McGill introduced the bars and framed rectangles experiment which judges the elementary perceptual task of position along non-aligned scales~\cite{cleveland_mcgill}. 

\subsection{Hypotheses}

We proposed two hypotheses entering the elementary perceptual task experiment:

\begin{itemize}
	\item \textbf{H4.1} \textbf{Classifiers can leverage additional visual cues.} The original bar and framed rectangle experiment shows how visual cues aid humans in mapping graphical elements to quantitative variables. This should be the same for feed-forward neural networks since they are inspired by the visual system.
	\item \textbf{H4.2} \textbf{Weber's law can be transferred to computational perception.} Cleveland and McGill confirmed Weber's law based on the bar and framed rectangle experiment. For humans, the ability to perceive change within a distribution is proportional to the size of the initial distribution.
\end{itemize}

\begin{figure}[t]
	  \includegraphics[width=\linewidth]{figure12_overview}
  \caption{\textbf{Bars and Framed Rectangles Experiment.} Cleveland and McGill introduce the bars and framed rectangles experiment which measures the perceptual task of judging position along non-aligned scales. For humans, it is easier to decide which of two bars represent a larger height if a scale is introduced by adding framed rectangles (right). In this case, the right bar is heigher as visible with less free space when adding the frame. We evaluate whether such a visual aid also helps machines to perceive visually encoded quantities.}
	\label{fig:bars_and_framed_rectangles_experiment}
\end{figure}

\subsection{Weber-Fechner's Law}

As identified by Cleveland and McGill, the bar and framed rectangle experiment is closely related to Weber's law. This psychophysics law states that perceivable difference within a distribution is proportional to the initial size of the distribution. Weber's law goes hand-in-hand with Fechner's law. We conduct an additional experiment based on the original illustrations of the Weber-Fechner law to investigate whether this law can be applied to computational perception of our classifiers (Fig.~\ref{fig:webers_law}).

\begin{figure}[t]
	  \includegraphics[width=\linewidth]{weber_overview}
  \caption{\textbf{Weber-Fechner Law.} The Weber-Fechner law states that the perceivable differences within a distribution is proportional to the initial size of the distribution. The lower square contains 10 more dots than the upper one on both sides. However, the difference is easily perceivable on the left while the squares on the right almost look the same. We generate rasterized visualizations similar to this setup and evaluate our classifiers.}
	\label{fig:webers_law}
\end{figure}

\subsection{Results}

First run indicates that framed rectangles perform better but we dont really know it yet.

\begin{figure*}[t]
	  \includegraphics[width=\linewidth]{figure12_mlae.pdf}
  \caption{\textbf{Computational results of the Bars-and-Framed-Rectangles experiment.} Log absolute error means and 95\% confidence intervals for the \emph{bars-and-framed-rectangles experiment} as described by Cleveland and McGill~\cite{cleveland_mcgill}. We test the performance of a Multi-layer Perceptron (MLP), the LeNet Convolutional Neural Network, as well as feature generation using the VGG19 and Xception networks trained on ImageNet.}
	\label{fig:figure12_mlae}
\end{figure*}



\section{Results and Discussion}

General discussion..

\subsection{Classifiers}

\textbf{Transfer Learning using ImageNet.} Classifiers trained on imagenet are tuned towards natural images. While VGG19 and Xception perform better than the shallower LeNet, their full performance only develops when training from scratch. This shows how natural images are truely different than infographics.
\\~\\
\textbf{Anti-aliasing.} Does it help? Not sure yet!




%
%\subsection{Elementary Perceptual Tasks}
%
%
%\subsection{Position-Angle Experiment}
%
%
%\subsection{Position-Length Experiment}
%
%\subsection{Bars and Framed Rectangles Experiment}
%


%\begin{figure}[t]
%	  \includegraphics[width=\linewidth]{figure12_val_loss.png}
%  \caption{\textbf{Classifier Efficiency of the Bars and Framed Rectangles experiment.} Categorical Cross-Entropy loss for the \emph{bars and framed rectangles experiment} as described by Cleveland and McGill~\cite{cleveland_mcgill}. The frame around the bars adds an additional visual cue enables faster network convergence. This is not yet reproducible!}
%	\label{fig:figure12_val_loss}
%\end{figure}


\section{Conclusions}

Future work: allow insights for infovis for machines



%% if specified like this the section will be committed in review mode
%\acknowledgments{
%The authors wish to thank A, B, and C. This work was supported in part by
%a grant from XYZ (\# 12345-67890).}

%\bibliographystyle{abbrv}
\bibliographystyle{abbrv-doi}
%\bibliographystyle{abbrv-doi-narrow}
%\bibliographystyle{abbrv-doi-hyperref}
%\bibliographystyle{abbrv-doi-hyperref-narrow}

\bibliography{paper.bib}
\end{document}

